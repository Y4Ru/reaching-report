% allgem. Dokumentenformat
\documentclass[a4paper,12pt,headsepline]{scrartcl}
\input{latex_einstellungen/variablen}

% weitere Pakete
% Grafiken aus PNG Dateien einbinden
\usepackage{graphicx}
\usepackage{scrextend}
\usepackage{gensymb}
\usepackage[numbers]{natbib}
\graphicspath{ {./img/methods/}{./img/results/}{./img/intro/}{./img/task/}{./img/procedure/}} 

% Eurozeichen einbinden
\usepackage[right]{eurosym}

% Umlaute unter UTF8 nutzen
\usepackage[utf8]{inputenc}

% Zeichenencoding
\usepackage[T1]{fontenc}

\usepackage{lmodern}
\usepackage{fix-cm}

% floatende Bilder ermöglichen
%\usepackage{floatflt}

% mehrseitige Tabellen ermöglichen
\usepackage{longtable}

% Unterstützung für Schriftarten
%\newcommand{\changefont}[3]{ 
%\fontfamily{#1} \fontseries{#2} \fontshape{#3} \selectfont}

% Packet für Seitenrandabständex und Einstellung für Seitenränder
\usepackage{geometry}
\geometry{left=3.5cm, right=2cm, top=2.5cm, bottom=2cm}

% Paket für Boxen im Text
\usepackage{fancybox}

% bricht lange URLs "schoen" um
\usepackage[hyphens,obeyspaces,spaces]{url}

% Paket für Textfarben
\usepackage{color}

% Mathematische Symbole importieren
\usepackage{amssymb}

% auf jeder Seite eine Überschrift (alt, zentriert)
%\pagestyle{headings}

% erzeugt Inhaltsverzeichnis mit Querverweisen zu den Kapiteln (PDF Version)
\usepackage[bookmarksnumbered,pdftitle={\titleDocument},hyperfootnotes=false]{hyperref} 
%\hypersetup{colorlinks, citecolor=red, linkcolor=blue, urlcolor=black}
%\hypersetup{colorlinks, citecolor=black, linkcolor= black, urlcolor=black}

% neue Kopfzeilen mit fancypaket
\usepackage{fancyhdr} %Paket laden
\pagestyle{fancy} %eigener Seitenstil
\fancyhf{} %alle Kopf- und Fußzeilenfelder bereinigen
\fancyhead[L]{\nouppercase{\leftmark}} %Kopfzeile links
\fancyhead[C]{} %zentrierte Kopfzeile
\fancyhead[R]{\thepage} %Kopfzeile rechts
\renewcommand{\headrulewidth}{0.4pt} %obere Trennlinie
%\fancyfoot[C]{\thepage} %Seitennummer
%\renewcommand{\footrulewidth}{0.4pt} %untere Trennlinie

% für Tabellen
\usepackage{array}

% Runde Klammern für Zitate
%\usepackage[numbers,round]{natbib}

% Festlegung Art der Zitierung - Havardmethode: Abkuerzung Autor + Jahr

% Schaltet den zusätzlichen Zwischenraum ab, den LaTeX normalerweise nach einem Satzzeichen einfügt.
\frenchspacing

% Paket für Zeilenabstand
\usepackage{setspace}

% für Bildbezeichner
\usepackage{capt-of}

% für Stichwortverzeichnis
\usepackage{makeidx}

% für Listings
\usepackage{listings}
\lstset{numbers=left, numberstyle=\tiny, numbersep=5pt, keywordstyle=\color{black}\bfseries, stringstyle=\ttfamily,showstringspaces=false,basicstyle=\footnotesize,captionpos=b}
\lstset{language=java}

% Indexerstellung
\makeindex

% Abkürzungsverzeichnis
\usepackage[german]{nomencl}
\let\abbrev\nomenclature

% Abkürzungsverzeichnis LiveTex Version
\renewcommand{\nomname}{Abkürzungsverzeichnis}
\setlength{\nomlabelwidth}{.25\hsize}
\renewcommand{\nomlabel}[1]{#1 \dotfill}
\setlength{\nomitemsep}{-\parsep}
\makenomenclature
%\makeglossary

% Abkürzungsverzeichnis TeTEX Version
% \usepackage[german]{nomencl}
% \makenomenclature
% %\makeglossary
% \renewcommand{\nomname}{Abkürzungsverzeichnis}
% \setlength{\nomlabelwidth}{.25\hsize}
% \renewcommand{\nomlabel}[1]{#1 \dotfill}
% \setlength{\nomitemsep}{-\parsep}

% Disable single lines at the start of a paragraph (Schusterjungen)
\clubpenalty = 10000
% Disable single lines at the end of a paragraph (Hurenkinder)
\widowpenalty = 10000
\displaywidowpenalty = 10000

\begin{document}
% hier werden die Trennvorschläge inkludiert
\input{latex_einstellungen/trennung}

%Schriftart Helvetica
%\changefont{phv}{m}{n}


% Titelseite %
% das Papierformat zuerst
%\documentclass[a4paper, 11pt]{article}

% deutsche Silbentrennung
%\usepackage[ngerman]{babel}

% wegen deutschen Umlauten
%\usepackage[ansinew]{inputenc}

% hier beginnt das Dokument
%\begin{document}


\thispagestyle{empty}

%\begin{figure}[t]
% \includegraphics[width=0.6\textwidth]{abb/fh_koeln_logo}
%\end{figure}


\begin{verbatim}


\end{verbatim}

\begin{center}
\Large{Universität Tübingen}\\
\Large{- Mathematisch-Naturwissenschaftliche Fakultät -}\\
\end{center}


\begin{center}
\Large{Externes Laborpraktikum}
\end{center}
\begin{verbatim}




\end{verbatim}
\begin{center}
\doublespacing
\textbf{\LARGE{Affordance Judgements in Virtual Reality}}\\
\singlespacing
\begin{verbatim}

\end{verbatim}
\textbf{im Studiengang Kognitionswissenschaft}
\end{center}
\begin{verbatim}

\end{verbatim}
\begin{center}

\end{center}
\begin{verbatim}

\end{verbatim}
\begin{center}
\textbf{}
\end{center}
\begin{verbatim}






\end{verbatim}
\begin{flushleft}
\begin{tabular}{llll}
\textbf{Autor:} & & Marcel Bechtold& \\
& & MatNr. 3949100 & \\
& & \\
\textbf{Version vom:} & & \today &\\
& & \\
\textbf{1. Betreuerin:} & & Prof. Dr. Martin Butz &\\
\textbf{2. Betreuer:} & & Dr. Betty Mohler &\\
\end{tabular}
\end{flushleft}

% römische Numerierung
%\pagenumbering{arabic}

% 1.5 facher Zeilenabstand
\onehalfspacing

% einfacher Zeilenabstand
\singlespacing

% Inhaltsverzeichnis anzeigen
\newpage
\tableofcontents

\newpage

% das Abbildungsverzeichnis
%\newpage
% Abbildungsverzeichnis soll im Inhaltsverzeichnis auftauchen
\addcontentsline{toc}{section}{List of Figures}
% Abbildungsverzeichnis endgueltig anzeigen
\listoffigures

% das Tabellenverzeichnis
%\newpage
% Abbildungsverzeichnis soll im Inhaltsverzeichnis auftauchen
\addcontentsline{toc}{section}{List of Tables}
% \fancyhead[L]{Abbildungsverzeichnis / Abkürzungsverzeichnis} %Kopfzeile links
% Abbildungsverzeichnis endgueltig anzeigen
\listoftables

% Definiert Stegbreite bei zweispaltigem Layout
\setlength{\columnsep}{25pt}

%%%%%%% EINLEITUNG %%%%%%%%%%%%
%\twocolumn
\newpage
\fancyhead[L]{\nouppercase{\leftmark}} %Kopfzeile links

% 1,5 facher Zeilenabstand
\onehalfspacing

% einzelne Kapitel
\newpage
\section{Introduction} \label{sec:introduction}

The goal of the internship is to learn how to implement and run Virtual Reality experiments. An existing experiment (Reaching Study) will be used to learn how to set up and run participants. Depending on how much time will be left two more follow-up experiments can be implemented and run during the internship.

In the virtual environment of the Reaching Study the participants find themselves sitting at a round table. Their task is to judge if they can reach for an object that appears on that table at different distances. Besides from their own perspective at the table most of the trials demand the participant to imagine to sit at a different location around the table. The question being addressed here is if the participants show different reaction times for different locations around the table which would argue for some kind of “mental rotation of the self in 3D”. Furthermore the data of these trials will provide the participants estimated maximum reaching distance. 
In the second part of the experiment the participant can actually reach for several objects on the table with a virtual arm controlled by the participant’s actual arm which is tracked by a controller. The virtual arm for one half of the participants is (20%) shorter and the other half (20%) longer than their actual arm.  
In the third part of the experiment the participants do the same task as in the first part. The question being addressed now is if the experience of a shorter/longer arm has the effect that the estimated maximum reaching distance is decreased/increased. This would support the idea that temporarily changing one’s body (and therefore one’s action capabilities) also changes perception. After having experienced a shorter (longer) arm, we argue the participants scale the object distance based on the new so called perceptual ruler which makes them perceive objects further away (closer). And as a result the estimated maximum reaching distance decreases (increases).
\newpage
\section{Methods}

\subsection{Participants}
37 participants(21 female, 16 male; all right-handed) from the local university community participated in the experiment. Their age ranged from 21 to 41 years. All participants were naive to the purpose of the experiment and had normal or corrected-to-normal vision. The experiment was approved by the ethics committee of the University of T\"ubingen, and was performed in accordance with the Declaration of Helsinki. Participants gave written informed consent prior to the experiment and were compensated with 8 Euro per hour for their participation. 

\subsection{Apparatus}
The virtual environment was displayed in stereo using an HTC Vive head-mounted-display (HMD) with a resolution of 1080 x 1200 pixels per eye (2160 x 1200 pixels combined). Inter-pupillary distance was measured with a pupilometer and set accordingly on the HTC Vive for each participant. Before the experiment two HTC Vive controllers were used to calibrate the experiment according to the participant's arm span. During the experiment one HTC Vive controller was attached to the participant's wrist to track hand and arm movements. The tracked controller movement data was used to display a virtual arm that moves according to the participants real movements. Participants were sitting during the whole experiment and viewed their virtual task in front of them. An XBox controller was used in order to allow participants to make decisions and proceed to the next trial.

In one part of the experiment (trial phase) the virtual environment consists of a round table and an empty chair which is located at the table and a small blue ball is located on top of the table. In the other part of the experiment (adaptation phase) the virtual environment consists of a square table an there a three different blue objects (a ball, a square and a capsule) located on top of the table. 


In the trial phase participants make a decision if they can reach the blue ball by pressing the shoulder buttons of the XBox Controller. The left shoulder button is used to decide that they can not reach the ball and the right shoulder button is used to decide they can reach it. Participants can start each trial by pressing the 'A' button of the XBox controller. In the adaptation phase participants only use their own hand movements to control their virtual arm which is tracked by the HTC Vive controller.

\begin{figure}[h]
\centering
\begin{subfigure}{.5\textwidth}
  \includegraphics[width=0.5\textwidth]{xbox}
  \caption{XBox controller used to make decisions in the trial phase.} 
  \label{fig:xbox}
\end{subfigure}%
\begin{subfigure}{.5\textwidth}
  \includegraphics[width=0.5\textwidth]{controller_attach}
  \caption{HTC Vive controller attached to the participant's arm to track hand and arm movements.}
  \label{fig:controller_attach}
\end{subfigure}%
\end{figure}

\begin{figure}[h]
\centering
\includegraphics[width=0.5\textwidth]{controller_attach}
\caption{HTC Vive controller attached to the participant's arm to track hand and arm movements.}
\label{fig:controller_attach}
\end{figure}

\begin{figure}[h]
\centering
\includegraphics[width=0.5\textwidth]{controller_calibration}
\caption{Two HTC Vive controller attached to the participant's arms to measure and calibrate the virtual arm.}
\label{fig:controller_attach}
\end{figure}

\begin{figure}[h]
\centering
\includegraphics[width=0.5\textwidth]{controller_use}
\caption{HTC Vive controller in use during the adaptation phase. Participants can control the virtual arm with their own hand and arm movements.}
\label{fig:controller_attach}
\end{figure}

\newpage

\subsection{Procedure}
Participants were invited as a group of two people and did not know each other. In one group session the two participants solved the task on their own in a separate virtual environment and together in the same virtual environment. The sequence of single and collaborative condition was counter balanced among all groups. In both the single and collaborative condition participants solved 20 tasks of which all had the same design as described above and differed only in color. In the single condition participants viewed the task always from the same perspective, which means that the starting cube was always placed at the same position in the solution space.
In the collaborative condition we rotated the solution space after 10 trials, which means the starting cube is on one participant's side for the first 10 trials and on the other participant's side for the remaining 10 trials. 

The detailed procedure for the single condition will be explained in the following:
After having read the instructions of the task participants also received a verbal instruction by the experimenter. Before the actual 20 trials began participants had to solve 4 training trials in order to verify that the participant understood the task and the way it has to be solved. It was emphasized that it is important to always stick to the sequence of putting in and removing cubes as described above. Furthermore participants could experience that the starting cube can not be removed and is the only cube in the experiment that collides with the other cubes (that was important to prevent other cubes from overlapping with the starting cube). Participants could also get used to the fact that the top color is always white and the bottom color is always black and that gray always faces to the inside of the solution space. The training trials could be started by the participant by holding a controller button for 2 seconds. When they started the training they saw the solution space surrounded by the 9 cubes which are all in reaching distance. They also saw the starting cube already being placed into the solution space. Above the task participants could see in which trial they currently are. In this case they would see "Training 1/4". Once they have finished a trial they can proceed with the next trial by pressing the controller button again for 2 seconds. 

Before the actual 20 trials started the experimenter repeated the most important instructions. Participants should try to solve the task as quickly and as accurately as possible. Furthermore they were not allowed to move around the problem space and should stay mostly stationary in their position with only a few steps to the sides allowed.
After the 4 training trials participants could start the actual trials as soon as they were ready by clicking a controller button again for 2 seconds. For the actual trial the experimenter would emphasize that participants should always make sure their solution is correct before proceeding with the next task. In case participants did not correctly solve a task and still proceed with the next task the experimenter took notes in order to exclude the trial from the analysis.

The procedure for the collaborative solving of the task was mostly the same. The only difference was that just one participant was able to skip to the next trial and therefore the instructions were that participants had to agree on when to proceed with the next task.

%what needs to go in here is:

%instructions that were given to the participants
%training phase (how many trials...)
%counterbalance single / multi (counterbalanced - 10 each side)

\newpage
\section{Results and Discussion}

Response times and accuracy was considered for analysis. Accuracy was calculated as the ratio of estimated to actual reach. Estimated maximum reach was determined by the crossover point from yes to no responses. (Fig. \ref{fig:cross_over}) The actual reach was measured after the experiment. A crossover point greater than 1 means over estimation, a crossover point smaller than 1 means under estimation. 

\begin{figure}
\centering
  \includegraphics[width=0.75\textwidth]{cross_over}
  \caption{Crossover point: ratio of estimated reach to actual reach.} 
  \label{fig:cross_over}
\end{figure}

\subsection{Response Time}

Firstly, response time did not significantly increase with a greater rotation of the chair. This is not inline with our hypothesis that participants show different reaction times for different locations around the table. Therefore we conclude that there is no cost in reaction time because of “mental rotation of the self in 3D”. An explanation for the consistent reaction times for different rotations could be a visual heuristic used by participants. It is possible that participants imagine a circle on the table which indicates how far they assume they are able to reach. This would make it unnecessary to actively rotate oneself to the target location so that reaction times are independent from the angle of the rotation.

Secondly, as shown in Fig. \ref{fig:no_arm_responst_time_dist} response times in the no arm condition were overall higher than in the shorter/longer arm condition. Since the no arm condition was always the first of the two trial phases this result could be caused by a training effect. Participants were less experienced with the task and therefore required more time to respond. Furthermore they have not yet had any control over their virtual arm so that their judgment is solely based on their imagination.  After having controlled their virtual arm in the adaptation phase, participants get a better understanding of how far they can reach in the second trial phase. This could also cause shorter reaction times in the second trial phase (shorter or longer arm condition).

\begin{figure}
\centering
  \includegraphics[width=0.75\textwidth]{all_response_time}
  \caption{Response times of different chair angles around the table in all conditions} 
  \label{fig:all_response_time}
\end{figure}

For each condition there is a clear peak of response times in relation to a specific distance of the ball from the edge of the table. The overall shape of the graph can be described as an upside down 'v' and therefore showing lower response times the shorter and higher distances get. In the no arm condition the peak is at 45 cm from the edge of the table with a response time of 2.5 seconds. (Fig \ref{fig:no_arm_responst_time_dist}) The peak response times in this graph indicate that it was the most difficult to make a judgment at a certain distance from the edge of the table. We assume that this distance must be very close to the participant's estimated maximum reaching distance. The closer the ball is to the participant's maximum reaching distance the more difficult it is to make a decision, because a yes and no response become equally likely.
Looking at shorter and higher distances one can observe the opposite effect of lower response times. Since it is obvious that an object can be reached when it is very near (or can not be reached when it is very far) it also easier to make a decision which results in lower response times. 

In the shorter arm condition the peak is at 30 cm from the edge of the table with a response time of 2.1 seconds. And in the longer arm condition the peak is at 40 cm with a response time of 1.7 seconds. (Fig \ref{fig:short_long_arm_responst_time_dist}) Expectedly, in the shorter arm condition participants show a peak reaction time at a shorter distance of the ball from the edge of the table than in the longer arm condition. Thus, having perceived and controlled their virtual arm not only results in shorter reaction times. It also affects the estimated maximum reaching distance. Having Perceived and controlled a shorter arm causes participants to experience a shorter maximum reaching distance and vise versa. 

Comparing the distances of shorter/longer arm condition at peak reaction times (30cm and 40cm) with the no arm condition it is interesting to observe that both the distances of shorter and longer arm condition are smaller than in the no arm condition (45cm). Without the data provided in the experiment one would probably assume that the distances must be higher for the the longer arm condition and smaller for the shorter arm condition. A general explanation for this effect might be the fact that participants are inclined to overestimate their maximum reaching distance without having perceived and controlled the virtual arm. Even if participants were told they should make a decision based on the fact that they can not lean forward, we assume they still incorporate such an increase in range of motion in their judgments, which leads to the overestimation. The opposite effect occurs after having experienced they virtual arm, because they have now experienced that their virtual arm will not reach further when they lean forward. That is because the virtual arm has a fixed shoulder position which is calculated according to the participant's actual bodily dimensions. Virtual arm movements are only animated and deducted by the controller movements (controllers attached to participants wrists). Also having perceived and controlled the virtual arm allows participants to get a visual ruler and therefore a more realistic perception of how far they are able to reach. Both the acceptance that they can not lean forward and the visual ruler provided in the adaptation phase are possible explanations for the decreased estimated maximum reaching distances in both the shorter and longer arm conditions.

Comparing the peak reaction times in the shorter and longer arm condition (1.7 seconds vs. 2.1 seconds) the peak reaction times in the shorter arm condition are significantly higher than in the longer arm condition. We assume that experiencing a shorter arm makes participants feel less comfortable and confident about decisions even if they do not know that the arm was shorter. Those assumptions needed to be verified with confidential votes in possible future replications or follow-up studies.

\begin{figure}
\centering
  \includegraphics[width=0.75\textwidth]{no_arm_responst_time_dist}
  \caption{Response times in relation to the edge of the table in the no arm condition.} 
  \label{fig:no_arm_responst_time_dist}
\end{figure}

\begin{figure}
\centering
  \includegraphics[width=0.75\textwidth]{short_long_arm_responst_time_dist}
  \caption{Response times in relation to the edge of the table in the shorter/longer arm condition.} 
  \label{fig:short_long_arm_responst_time_dist}
\end{figure}

\subsection{Accuracy}

In the no arm condition accuracy (crossover point) ranged from 1.15 to 0.9 . The highest estimation was at 0 degrees. The lowest estimation was at 135 degrees. (Fig. \ref{fig:no_arm_co})

\begin{figure}
\centering
  \includegraphics[width=0.75\textwidth]{no_arm_co}
  \caption{Accuracy (crossover point) of different chair angles around the table in the no arm condition} 
  \label{fig:no_arm_co}
\end{figure}

 In the shorter arm condition accuracy (crossover point) ranged from 1.1 to 0.85. The highest estimation was at 0 degrees. The lowest estimation was at 90 degrees. In the longer arm condition accuracy (crossover point) ranged from 0.66 to 0.82 . The highest estimation was at 0 degrees. The lowest estimation was at 135 degrees. (\ref{fig:short_long_arm_co}) 
 
 In all conditions the lowest estimations were between 90 to 135 degrees. This means most low estimations have occured for trials in which the ball was on the left half of the rounds table (degrees were defined clockwise around the table with 0 degrees at the participant's actual location). Since all participants were right-handed their judgments may have been less confident for trials in which the ball was in the left half of the table. 
 
 An general interaction between degree around the table and accuracy could only be found for 0 degrees, because in all conditions the estimation at 0 degrees are the highest. This is probably due to the fact that confidence is higher in the first person perspective. No significant effect could be found for the opposite position at 180 degrees. Also for all other degrees there was no significant interaction because such effects need to occur pairwise (45/215, 90/270, 135/225) which was not the case.
 
 Participants in both the shorter and longer arm condition show overall lower estimations than in the now arm condition, which is inline with the observation of reaction times. 
 

 
 \begin{figure}
\centering
  \includegraphics[width=0.75\textwidth]{short_long_arm_co}
  \caption{Accuracy (crossover point) of different chair angles around the table in the shorter/longer arm condition} 
  \label{fig:short_long_arm_co}
\end{figure}

\newpage
\clearpage
\section{Discussion}\label{discussion}

2. Follow-up: affordance judgments for others

The first follow-up experiment will be changed only in the way that participants will be asked to not imagine themselves sitting at a different location, but to judge for an avatar appearing at different locations around the table if it would be possible for him/her to reach the object. 
The main question then is if participants - after having experienced a shorter/longer virtual arm – also show an increase/decrease in estimated maximum reaching distance for the avatar. If so we would argue that temporary changes to our body not only influence how we perceive our own action capabilities but also how we perceive action capabilities of others.

3. Follow up: 3rd person adaptation

The second follow-up experiment will only change the participant’s perspective in the second part of the experiment from 1st to 3rd person perspective. This means that the participant’s own arm movements now control the arm of an avatar in the scene instead of the participant’s own virtual arm (the participant’s own arm will not be rendered at all). 
With this manipulation we want to address the question if the experience of controlling the arm of an avatar will also alter the participant’s judgement of how far he/she is able to reach. We argue that in contrast to the initial study with this manipulation the participant’s perceptual ruler is not changed and therefore expect little to no change in estimated maximum reaching distance.
\newpage
\section{Conclusion}\label{conclusion}
Finally, it can be said that the approach presented in this study can be considered as a basis for future research to further investigate the relevant cognitive processes involved in collaborative spatial problem solving. This study was intentionally designed to be very controlled and therefore does not compare to any application of real life problem solving. The goal was to design a task that allows to quantify performance in spacial problem solving tasks. Also the study proved virtual reality to be a well suited method of research for the field of collaborative spatial problem solving. The main finding was that groups perform significantly better than individuals in the designed spatial problem solving task. Further research will have to show if the same applies for more complex variations of the task and how important factors like perspective and communication are.

\onecolumn
% einfacher Zeilenabstand
\singlespacing
% Literaturliste soll im Inhaltsverzeichnis auftauchen
\newpage
\addcontentsline{toc}{section}{Bibliography}
% Literaturverzeichnis anzeigen
\renewcommand\refname{Bibliography}
\bibliographystyle{plainnat}
\bibliography{Hauptdatei}

%% Index soll Stichwortverzeichnis heissen
% \newpage
% % Stichwortverzeichnis soll im Inhaltsverzeichnis auftauchen
% \addcontentsline{toc}{section}{Stichwortverzeichnis}
% \renewcommand{\indexname}{Stichwortverzeichnis}
% % Stichwortverzeichnis endgueltig anzeigen
% \printindex

\onehalfspacing
% evtl. Anhang

% Eidesstattliche Erklärung
\include{erklaerung}

\end{document}