\section{Introduction} \label{sec:introduction}

The goal of the internship is to learn how to implement and run Virtual Reality experiments. An existing experiment (Reaching Study) will be used to learn how to set up and run participants. Depending on how much time will be left two more follow-up experiments can be implemented and run during the internship.

In the virtual environment of the Reaching Study the participants find themselves sitting at a round table. Their task is to judge if they can reach for an object that appears on that table at different distances. Besides from their own perspective at the table most of the trials demand the participant to imagine to sit at a different location around the table. The question being addressed here is if the participants show different reaction times for different locations around the table which would argue for some kind of “mental rotation of the self in 3D”. Furthermore the data of these trials will provide the participants estimated maximum reaching distance. 
In the second part of the experiment the participant can actually reach for several objects on the table with a virtual arm controlled by the participant’s actual arm which is tracked by a controller. The virtual arm for one half of the participants is (20%) shorter and the other half (20%) longer than their actual arm.  
In the third part of the experiment the participants do the same task as in the first part. The question being addressed now is if the experience of a shorter/longer arm has the effect that the estimated maximum reaching distance is decreased/increased. This would support the idea that temporarily changing one’s body (and therefore one’s action capabilities) also changes perception. After having experienced a shorter (longer) arm, we argue the participants scale the object distance based on the new so called perceptual ruler which makes them perceive objects further away (closer). And as a result the estimated maximum reaching distance decreases (increases).