\section{Conclusion}\label{conclusion}

Need to assess strategy participants, instructions obviously not followed?

training effect

measurement accuracy of arm

80/120 vs 90/110

Based on the proposed improvements two follow up studies could provide further insights.

2. Follow-up: affordance judgments for others

The first follow-up experiment will be changed only in the way that participants will be asked to not imagine themselves sitting at a different location, but to judge for an avatar appearing at different locations around the table if it would be possible for him/her to reach the object. 
The main question then is if participants - after having experienced a shorter/longer virtual arm – also show an increase/decrease in estimated maximum reaching distance for the avatar. If so we would argue that temporary changes to our body not only influence how we perceive our own action capabilities but also how we perceive action capabilities of others.

3. Follow up: 3rd person adaptation

The second follow-up experiment will only change the participant’s perspective in the second part of the experiment from 1st to 3rd person perspective. This means that the participant’s own arm movements now control the arm of an avatar in the scene instead of the participant’s own virtual arm (the participant’s own arm will not be rendered at all). 
With this manipulation we want to address the question if the experience of controlling the arm of an avatar will also alter the participant’s judgement of how far he/she is able to reach. We argue that in contrast to the initial study with this manipulation the participant’s perceptual ruler is not changed and therefore expect little to no change in estimated maximum reaching distance.